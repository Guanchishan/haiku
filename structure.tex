%!TEX root = haiku.tex

%----------------------------------------------------------------------------------------
%	PAPER, MARGIN AND HEADER/FOOTER SIZES
%----------------------------------------------------------------------------------------

\setstocksize{13cm}{9cm} % Paper size
\settrimmedsize{\stockheight}{\stockwidth}{*} % No trims
\setlrmarginsandblock{18pt}{18pt}{*} % Left/right margins
\setulmarginsandblock{30pt}{36pt}{*} % Top/bottom margins
\setheadfoot{14pt}{12pt} % Header/footer height
\setheaderspaces{*}{8pt}{*} % Extra header space

%----------------------------------------------------------------------------------------
%	FOOTNOTE CUSTOMIZATION
%----------------------------------------------------------------------------------------

\renewcommand{\foottextfont}{\itshape\footnotesize} % Font settings for footnotes
\setlength{\footmarkwidth}{-.1em} % Space between the footnote number and the text
\setlength{\footmarksep}{.1em} % Space between multiple footnotes on the same page
\renewcommand*{\footnoterule}{} % Remove the rule above the first footnote
\setlength{\skip\footins}{1\onelineskip} % Space between the body text and the footnote

%----------------------------------------------------------------------------------------
%	HEADER AND FOOTER FORMATS
%----------------------------------------------------------------------------------------

\makepagestyle{mio} % Define a new custom page style
\setlength{\headwidth}{\textwidth} % Header the same width as the text
\makeheadrule{mio}{\textwidth}{0.1mm} % Header rule height
\makeevenhead{mio}{\scriptsize\leftmark}{}{\scriptsize\rightmark}
\makeoddhead{mio}{\scriptsize\rightmark}{}{\scriptsize\leftmark}
\makeoddfoot{mio}{}{\scriptsize {\thepage}}{} % Footer specification
\makeoddfoot{plain}{}{\footnotesize {\thepage}}{} % Pages of chapters
\pagestyle{mio} % Set the page style to the custom style defined above

%----------------------------------------------------------------------------------------
%	PART FORMAT
%----------------------------------------------------------------------------------------

\renewcommand{\partnamefont}{\centering\sffamily\itshape\Huge} % Part name font specification
\renewcommand{\partnumfont}{\sffamily\Huge} % Part number font specification
\renewcommand{\parttitlefont}{\centering\sffamily\scshape} % Part title font specification
\renewcommand{\beforepartskip}{\null\vskip.618\textheight} % Whitespace above the part heading

%----------------------------------------------------------------------------------------
%	CHAPTER FORMAT
%----------------------------------------------------------------------------------------

\makechapterstyle{Tufte}{ % Define a new chapter style
    \renewcommand{\chapterheadstart}{\null \vskip3.5\onelineskip} % Whitespace before the chapter starts
    \renewcommand{\printchaptername}{\large\itshape\chaptername} % "Chapter" text font specification
    \renewcommand{\printchapternum}{\LARGE\thechapter \\} % Chapter number font specification
    \renewcommand{\afterchapternum}{} % Space between the chapter number and text
    \renewcommand{\printchaptertitle}[1]{ % Chapter title font specification
        \raggedright
        \itshape\Huge{##1}}
        \renewcommand{\afterchaptertitle}{
            \vskip3.5\onelineskip
        }}
        \chapterstyle{Tufte} % Set the chapter style to the custom style defined above

%----------------------------------------------------------------------------------------
%	SECTION FORMAT
%----------------------------------------------------------------------------------------

        \setsecheadstyle{\sethangfrom{\noindent ##1}\raggedright\sffamily\itshape\Large} % Section title font specification
        \setbeforesecskip{-.6\onelineskip} % Whitespace before the section
        \setaftersecskip{.3\onelineskip} % Whitespace after the section

%----------------------------------------------------------------------------------------
%	SUBSECTION FORMAT
%----------------------------------------------------------------------------------------

        \setsubsecheadstyle{\sethangfrom{\noindent  ##1}\raggedright\sffamily\large\itshape} % Subsection title font specification
        \setbeforesubsecskip{-.5\onelineskip} % Whitespace before the subsection
        \setaftersubsecskip{.2\onelineskip} % Whitespace after the subsection

%----------------------------------------------------------------------------------------
%	SUBSUBSECTION FORMAT
%----------------------------------------------------------------------------------------

        \setsubsubsecheadstyle{\sethangfrom{\noindent ##1}\raggedright\sffamily\itshape} % Subsubsection title font specification
        \setbeforesubsubsecskip{-.5\onelineskip} % Whitespace before the subsubsection
        \setaftersubsubsecskip{.1\onelineskip} % Whitespace after the subsubsection

%----------------------------------------------------------------------------------------
%	CAPTION FORMAT
%----------------------------------------------------------------------------------------

        \captiontitlefont{\itshape\footnotesize} % Caption font specification
        \captionnamefont{\footnotesize} % "Caption" text font specification

%----------------------------------------------------------------------------------------
%	QUOTATION ENVIRONMENT FORMAT
%----------------------------------------------------------------------------------------

        \renewenvironment{quotation}
        {\par\leftskip=1em\vskip.5\onelineskip\em}
        {\par\vskip.5\onelineskip}

%----------------------------------------------------------------------------------------
%	QUOTE ENVIRONMENT FORMAT
%----------------------------------------------------------------------------------------

        \renewenvironment{quote}
    {\list{}{\em\leftmargin=1em}\item[]}{\endlist\relax}

%----------------------------------------------------------------------------------------
%	MISCELLANEOUS DOCUMENT SPECIFICATIONS
%----------------------------------------------------------------------------------------

        \midsloppy % Fewer overfull lines - used in the memoir class and allows a setting somewhere between \fussy and \sloppy

        \checkandfixthelayout % Tell memoir to implement the above
