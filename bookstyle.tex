%!TEX root = haiku.tex

\setlength{\headheight}{15pt}
\newcounter{haikucounter}
\renewcommand{\thehaikucounter}{\arabic{haikucounter}}

\newenvironment{haiku}{
    \refstepcounter{haikucounter}
    \centering\thehaikucounter\\[1ex]\begin{parse lines}{##1\newline}
    }%
    {\end{parse lines}}

\setcounter{secnumdepth}{-1}
\setcounter{tocdepth}{1}

% \newfontfamily\fontfamilyname{Times}
% \setmainfont{Times}
\newfontfamily{\FS}[Scale=MatchLowercase, Path=fonts/]{NotoSerifCJKsc-Regular.otf}
% \newfontfamily{\FS}[Scale=MatchLowercase, Path=fonts/]{NotoSerifCJKsc-Regular.otf}
\newfontfamily{\FK}[Scale=MatchUppercase, Path=fonts/]{kaiti.ttf}
\newfontfamily{\FT}[Scale=MatchLowercase, Path=fonts/]{fangsong.otf}
\newfontfamily{\FF}[Path=fonts/]{HanyiSentyVimalkirti.ttf} % 汉仪新蒂维摩诘经体
% \newfontfamily{\FH}[Scale=MatchUppercase, Path=fonts/]{kyoukasho.otf} % 教科書体
\newfontfamily{\FH}[Scale=MatchUppercase, Path=fonts/]{NotoSerifCJKjp-SemiBold.otf}
\newfontfamily{\FM}[Scale=MatchUppercase, Path=fonts/]{NotoSerifCJKjp-Regular.otf}

\graphicspath{{figures/}}

\renewcommand{\rubysize}{0.5}
\renewcommand{\rubysep}{0em}
\renewcommand{\contentsname}{\FK 目录}
\renewcommand{\bookname}{}

\XeTeXlinebreaklocale "zh"
\XeTeXlinebreakskip = 0pt plus 1pt
\defaultfontfeatures{Mapping=tex-text}
% Chinese indent
\makeatletter
\let\@afterindentfalse\@afterindenttrue
\@afterindenttrue
\makeatother
\setlength{\parindent}{2em}

\setlength{\footnotemargin}{1em}
\renewcommand{\hangfootparskip}{0pt}
\renewcommand{\thefootnote}{\arabic{footnote}}

\definecolor{fondpaille}{RGB}{230,230,207}
\pagecolor{fondpaille}
